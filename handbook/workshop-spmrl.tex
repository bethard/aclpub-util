\sessionabstracts{Statistical Parsing of Morphologically-Rich Languages}{Program Chairs:
Mohammed Attia,
Bernd Bohnet,
Marie Candito,
Aoife Cahill,
\"Ozlem Cetinoglu,
Jinho D. Choi,
Grzegorz Chrupa\l{}a,
Benoit Crabb\'e,
G\"{u}l\c{s}en Cebiro\u{g}lu Eryi\u{g}it,
Michael Elhadad,
Richard Farkas,
Jennifer Foster,
Josef van Genabith,
Koldo Gojenola,
Spence Green,
Samar Husain,
Sandra K\"ubler,
Jonas Kuhn,
Alberto Lavelli,
Joseph Le Roux,
Wolfgang Maier,
Takuya Matsuzaki,
Joakim Nivre,
Kemal Oflazer,
Adam Przepi\'{o}rkowski,
Owen Rambow,
Kenji Sagae,
Benoit Sagot,
Djam\'{e} Seddah,
Reut Tsarfaty,
Lamia Tounsi,
Daniel Zeman}{}{
\sessionabstract{Friday}{9:10}{10:05}{Blewett Suite}{Invited talk: An HDP Model for Inducing CCGs}{Julia Hockenmaier}{}
\sessionabstractsep
\sessionabstract{Friday}{10:05}{10:30}{Blewett Suite}{Working with a small dataset - semi-supervised dependency parsing for Irish}{ Teresa Lynn,  Jennifer Foster,  Mark Dras}{We present a number of semi-supervised parsing experiments on the Irish language carried out using a small seed set of manually parsed trees  and a larger, yet still relatively small, set of unlabelled sentences. We take two popular dependency parsers -- one graph-based and one transition-based -- and compare results for both. Results show that using semi-supervised learning in the form of self-training and co-training  yields only very modest improvements in parsing accuracy. We also try to  use morphological information in a targeted way and fail to see any improvements.}
\sessionabstractsep
\sessionabstract{Friday}{11:00}{11:25}{Blewett Suite}{Lithuanian Dependency Parsing with Rich Morphological Features}{ Jurgita Kapociute-Dzikiene,  Joakim Nivre,  Algis Krupavicius}{We present the first statistical dependency parsing results for Lithuanian, a morphologically rich language in the Baltic branch of the Indo-European family. Using a greedy transition-based parser, we obtain a labeled attachment score of 74.7 with gold morphology and 68.1 with predicted morphology (77.8 and 72.8 unlabeled). We investigate the usefulness of different features and find that rich morphological features improve parsing accuracy significantly, by 7.5 percentage points with gold features and 5.6 points with predicted features. As expected, CASE is the single most important morphological feature, but virtually all available features bring some improvement, especially under the gold condition.}
\sessionabstractsep
\sessionabstract{Friday}{11:25}{11:50}{Blewett Suite}{Parsing Croatian and Serbian by Using Croatian Dependency Treebanks}{ \v{Z}eljko Agi\'{c},  Danijela Merkler,  Da\v{s}a Berovi\'{c}}{We investigate statistical dependency parsing of two closely related languages, Croatian and Serbian. As these two morphologically complex languages of relaxed word order are generally under-resourced -- with the topic of dependency parsing still largely unaddressed, especially for Serbian -- we make use of the two available dependency treebanks of Croatian to produce state-of-the-art parsing models for both languages. We observe parsing accuracy on four test sets from two domains. We give insight into overall parser performance for Croatian and Serbian, impact of preprocessing for lemmas and morphosyntactic tags and influence of selected morphosyntactic features on parsing accuracy.}
\sessionabstractsep
\sessionabstract{Friday}{11:50}{12:15}{Blewett Suite}{A Cross-Task Flexible Transition Model for Arabic Tokenization, Affix Detection, Affix Labeling, POS Tagging, and Dependency Parsing}{ Stephen Tratz}{This paper describes cross-task flexible transition models (CTF-TMs) and demonstrates their effectiveness for Arabic natural language processing (NLP). NLP pipelines often suffer from error propagation, as errors committed in lower-level tasks cascade through the remainder of the processing pipeline. By allowing a flexible order of operations across and within multiple NLP tasks, a CTF-TM can mitigate both cross-task and within-task error propagation. Our Arabic CTF-TM models tokenization, affix detection, affix labeling, part-of-speech tagging, and dependency parsing, achieving state-of-the-art results. We present the details of our general framework, our Arabic CTF-TM, and the setup and results of our experiments.}
\sessionabstractsep
\sessionabstract{Friday}{14:10}{14:20}{Blewett Suite}{The LIGM-Alpage architecture for the SPMRL 2013 Shared Task: Multiword Expression Analysis and Dependency Parsing}{ Matthieu Constant,  Marie Candito,  Djam\'{e} Seddah}{This paper describes the LIGM-Alpage system for the SPMRL 2013 Shared Task. We only participated to the French part of the dependency parsing track, focusing on the realistic setting where the system is informed neither with gold tagging and morphology nor (more importantly) with gold grouping of tokens into multi-word expressions (MWEs). While the realistic scenario of predicting both MWEs and syntax has already been investigated for constituency parsing, the SPMRL 2013 shared task datasets offer the possibility to investigate it in the dependency framework. We obtain the best results for French, both for overall parsing and for MWE recognition, using a reparsing architecture that combines several parsers, with both pipeline architecture (MWE recognition followed by parsing), and joint architecture (MWE recognition performed by the parser).}
\sessionabstractsep
\sessionabstract{Friday}{14:20}{14:30}{Blewett Suite}{Exploring beam-based shift-reduce dependency parsing with DyALog: Results from the SPMRL 2013 shared task}{ Eric De La Clergerie}{The SPMRL 2013 shared task was the opportunity to develop and test,   with promising results, a simple beam-based shift-reduce dependency   parser on top of the tabular logic programming system DyALog. The   parser was also extended to handle ambiguous word lattices, with   almost no loss w.r.t. disambiguated input, thanks to specific   training, use of oracle segmentation, and large beams. We believe   that this result is an interesting new one for shift-reduce parsing.}
\sessionabstractsep
\sessionabstract{Friday}{14:30}{14:55}{Blewett Suite}{Effective Morphological Feature Selection with MaltOptimizer at the SPMRL 2013 Shared Task}{ Miguel Ballesteros}{The inclusion of morphological features provides very useful information that helps to enhance the results when parsing morphologically rich languages. MaltOptimizer is a tool, that given a data set, searches for the optimal parameters, parsing algorithm and optimal feature set achieving the best results that it can find for parsers trained with MaltParser. In this paper, we present an extension of MaltOptimizer that explores, one by one and in combination, the features that are geared towards morphology. From our experiments in the context of the Shared Task on Parsing Morphologically Rich Languages, we extract an in-depth study that shows which features are actually useful for transition-based parsing and we provide competitive results, in a fast and simple way.}
\sessionabstractsep
\sessionabstract{Friday}{14:55}{15:10}{Blewett Suite}{Exploiting the Contribution of Morphological Information to Parsing: the BASQUE TEAM system in the SPRML'2013 Shared Task}{ Iakes Goenaga,  Koldo Gojenola,  Nerea Ezeiza}{This paper presents a dependency parsing system, presented as BASQUE\_TEAM at the SPMRL'2013 Shared Task, based on the analysis of each morphological feature of the languages. Once the specific relevance of each morphological feature is calculated, this system uses the most significant of them to create a series of analyzers using two freely available and state of the art dependency parsers, MaltParser and Mate. Finally, the system will combine previously achieved parses using a voting approach.}
\sessionabstractsep
\sessionabstract{Friday}{15:10}{15:20}{Blewett Suite}{The AI-KU System at the SPMRL 2013 Shared Task :  Unsupervised Features for Dependency Parsing}{ Volkan Cirik,  H\"{u}sn\"{u} \c{S}ensoy}{We propose the use of the word categories and embeddings induced from raw text as auxiliary features in dependency parsing. To induce word features, we make use of contextual, morphologic and orthographic properties of the words. To exploit the contextual information, we make use of substitute words, the most likely substitutes for target words, generated by using a statistical language model. We generate morphologic and orthographic properties of word types in an unsupervised manner. We use a co-occurrence model with these properties to embed words onto a 25-dimensional unit sphere. The AI-KU system shows improvements for some of the languages it is trained on for the first Shared Task of Statistical Parsing of Morphologically Rich Languages.}
\sessionabstractsep
\sessionabstract{Friday}{15:20}{15:30}{Blewett Suite}{SPMRL'13 Shared Task System: The CADIM Arabic Dependency Parser}{ Yuval Marton,  Nizar Habash,  Owen Rambow,  Sarah Alkhulani}{We describe the submission from the Columbia Arabic \& Dialect Modeling group (CADIM) for the Shared Task at the Fourth Workshop on Statistical Parsing of Morphologically Rich Languages (SPMRL'2013). We participate in the Arabic Dependency parsing task for predicted POS tags and features. Our system is based on Marton et al. (2013).}
\sessionabstractsep
\sessionabstract{Friday}{16:00}{16:20}{Blewett Suite}{A Statistical Approach to Prediction of Empty Categories in Hindi Dependency Treebank}{ Puneeth Kukkadapu,  Prashanth Mannem}{In this paper we use statistical dependency parsing techniques to detect NULL or Empty categories in the Hindi sentences. We have currently worked on Hindi dependency treebank which is released as part of COLING-MTPIL 2012 Workshop. Earlier Rule based approaches are employed to detect Empty heads for Hindi language but statistical learning for automatic prediction is not explored. In this approach we used a technique of introducing complex labels into the data to predict Empty Categories in sentences. We have also discussed about shortcomings and difficulties in this approach and evaluated the performance of this approach on different Empty categories.}
\sessionabstractsep
\sessionabstract{Friday}{16:00}{16:20}{Blewett Suite}{An Empirical Study on the Effect of Morphological and Lexical Features in Persian Dependency Parsing}{ Mojtaba Khallash,  Ali Hadian,  Behrouz Minaei-Bidgoli}{This paper investigates the impact of different morphological and lexical information on data-driven dependency parsing of Persian, a morphologically rich language. We explore two state-of-the-art parsers, namely MSTParser and MaltParser, on the recently released Persian dependency treebank and establish some baselines for dependency parsing performance. Three sets of issues are addressed in our experiments: effects of using gold and automatically derived features, finding the best features for the parser, and a suitable way to alleviate the data sparsity problem. The final accuracy is 87.91\% and 88.37\% labeled attachment scores for MaltParser and MSTParser, respectively.}
\sessionabstractsep
\sessionabstract{Friday}{16:00}{16:20}{Blewett Suite}{Constructing a Practical Constituent Parser from a Japanese Treebank with Function Labels}{ Takaaki Tanaka,  Masaaki NAGATA}{We present an empirical study on constructing a Japanese constituent parser, which can output function labels  to deal with more detailed syntactic information. Japanese syntactic parse trees are usually represented as unlabeled  dependency structure between bunsetsu chunks,  however, such expression is insufficient to uncover the syntactic information about distinction between complements and adjuncts and coordination structure, which is required for practical applications such as  syntactic reordering of machine translation.  We describe a preliminary effort on constructing a Japanese constituent parser by a Penn Treebank style treebank semi-automatically made from  a dependency-based corpus. The evaluations show  the parser trained on the treebank has comparable bracketing accuracy as conventional bunsetsu-based parsers, and  can output such function labels  as the grammatical role of the argument and the type of adnominal phrases.}
\sessionabstractsep
\sessionabstract{Friday}{16:00}{16:20}{Blewett Suite}{Context Based Statistical Morphological Analyzer and its Effect on Hindi Dependency Parsing}{ Deepak Kumar Malladi,  Prashanth Mannem}{This paper revisits the work of (Malladi and Mannem, 2013) which focused on building a Statistical Morphological Analyzer (SMA) for Hindi and compares the performance of SMA with other existing statistical analyzer, Morfette. We shall evaluate SMA in various experiment scenarios and look at how it performs for unseen words. The later part of the paper presents the effect of the predicted morph features on dependency parsing and extends the work to other morphologically rich languages: Hindi and Telugu, without any language-specific engineering.}
\sessionabstractsep
\sessionabstract{Friday}{16:00}{16:20}{Blewett Suite}{Representation of Morphosyntactic Units and Coordination Structures in the Turkish Dependency Treebank}{ Umut Sulubacak,  G\"{u}l\c{s}en Eryi\u{g}it}{This paper presents our preliminary conclusions as part of an ongoing effort to construct a new dependency representation framework for Turkish. We aim for this new framework to accommodate the highly agglutinative morphology of Turkish as well as to allow the annotation of unedited web data, and shape our decisions around these considerations. In this paper, we firstly describe a novel syntactic representation for morphosyntactic sub-word units (namely inflectional groups (IGs) in Turkish) which allows inter-IG relations to be discerned with perfect accuracy without having to hide lexical information. Secondly, we investigate alternative annotation schemes for coordination structures and present a better scheme (nearly 11\% increase in recall scores) than the one in Turkish Treebank (Oflazer et al., 2003) for both parsing accuracies and compatibility for colloquial language.}
\sessionabstractsep
\sessionabstract{Friday}{16:20}{16:45}{Blewett Suite}{(Re)ranking Meets Morphosyntax: State-of-the-art Results from the SPMRL 2013 Shared Task}{ Anders Bj\"{o}rkelund,  Ozlem Cetinoglu,  Rich\'{a}rd Farkas,  Thomas Mueller,  Wolfgang Seeker}{This paper describes the IMS-SZEGED-CIS contribution to the SPMRL 2013 Shared Task. We participate in both the constituency and dependency tracks, and achieve state-of-the-art for all languages. For both tracks we make significant improvements through high quality preprocessing and (re)ranking on top of strong baselines. Our system came out first for both tracks.}
\sessionabstractsep
\sessionabstract{Friday}{16:45}{17:10}{Blewett Suite}{Invited talk: Dependency Parsing with Selectional Branching}{Jinho D. Choi}{}
\sessionabstractsep
\sessionabstract{Friday}{18:00}{18:15}{Blewett Suite}{Overview of the SPMRL 2013 Shared Task: A Cross-Framework Evaluation of Parsing Morphologically Rich Languages}{ Djam\'{e} Seddah,  Reut Tsarfaty,  Sandra K\"{u}bler,  Marie Candito,  Jinho D. Choi,  Rich\'{a}rd Farkas,  Jennifer Foster,  Iakes Goenaga,  Koldo Gojenola Galletebeitia,  Yoav Goldberg,  Spence Green,  Nizar Habash,  Marco Kuhlmann,  Wolfgang Maier,  Yuval Marton,  Joakim Nivre,  Adam Przepi\'{o}rkowski,  Ryan Roth,  Wolfgang Seeker,  Yannick Versley,  Veronika Vincze,  Marcin Woli\'{n}ski,  Alina Wr\'{o}blewska}{This paper reports on the first shared task on statistical parsing of Morphologically Rich Languages (MRLs). The task features data sets from nine languages, each available both in constituency and dependency annotation. We report on the alignment of the data sets, on the proposed parsing scenarios, and on the evaluation metrics for parsing MRLs given different representation types. We present and analyze parsing results obtained by the task participants, and then provide an analysis and cross- comparison of the parsers across languages and frameworks, reported for gold input as well as more realistic parsing scenarios.}
}
